\documentclass{article}
\usepackage[utf8]{inputenc}
\usepackage{amsmath}
\usepackage{amsthm}
\usepackage{amssymb}
\usepackage{enumitem}
\usepackage[left=2.5cm,right=2.5cm,top=2.25cm,bottom=2.25cm]{geometry}
\usepackage{multirow}

\title{LINMA2471 -- Optimization models ans methods II \\ Notes from the $6^\mathrm{th}$ lecture}
\author{Antoine Aspeel \and Pierre-Paul Mouchet \and Guillaume Olikier}
\date{October 21, 2015}

% Definitions and propositions
\newtheorem{theorem}{Theorem}
\newtheorem{prop}[theorem]{Proposition}
\newtheorem{lemma}[theorem]{Lemma}
\newtheorem{coro}[theorem]{Corollary}
\theoremstyle{definition}
\newtheorem{defi}{Definition}
\newtheorem{ex}{Example}

% Shortcuts
\newcommand{\myspan}{\mathrm{span}} % span
\newcommand{\myint}{\mathrm{int}} % span
\newcommand{\e}{\mathrm{e}} % e for exponential
\newcommand{\dd}{\mathrm{d}} % d differential
\newcommand{\J}{\mathrm{J}} % jacobian

\newcommand{\vnorm}[1]{\left\Vert #1 \right\Vert}

\begin{document}

\maketitle



\section{Reminder on solvers}

This section brings complements to what was discussed in the first lecture about solvers.

%\begin{tabular}{|c|c|c|}
%\hline
%Solvers & Problems & Integer variables \\
%\hline
%CPLEX & linear optimization & yes \\
%GUROBI & convex quadratic optimization & yes \\
%KNITRO, SNOPT, MINOS & nonlinear optimization & yes for KNITRO but loss of efficiency \\
%BARON & global optimization & no \\
%\hline
%\end{tabular}

\begin{tabular}{|c|c|c|}
\hline
Solvers & Problems & Integer variables \\
\hline
\multirow{2}{*}{CPLEX,GUROBI} & linear optimization and & \multirow{2}{*}{yes} \\
 & convex quadratic optimization &  \\[0.1cm]
KNITRO, SNOPT, MINOS & nonlinear optimization & yes for KNITRO but loss of efficiency \\[0.1cm]
BARON & global optimization & no \\
\hline
\end{tabular}

\section{Gradient method for unconstrained problems}

Let us recall basic definitions from last lecture.
\begin{defi}
Given $L>0$, we say $f : D \subseteq \mathbb{R}^n \to \mathbb{R}$ has \emph{$L$-Lipschitz gradient} if and only if $f \in C^1(D)$ and
\begin{equation*}
\vnorm{\nabla f(x) - \nabla f(y)} \le L \vnorm{x-y}
\end{equation*}
for all $x,y \in D$. We denote $C_L^{1,1}(D)$ the set of such functions. We also define
\begin{equation*}
F_L^{1,1}(D) = \{ f \in C_L^{1,1}(D) \mid f \text{ is convex}\} .
\end{equation*}
\end{defi}

\noindent Given $f \in C^1$, we denote $T^1_y(x) = f(y) + \nabla f(y)^T (x-y)$ the first-order Taylor expansion of $f$ around $y$ evaluated at point $x$.

\subsection{Gradient method for functions of $C_L^{1,1}$}

Last week, we studied the gradient method for the general problem
\begin{equation*}
\min_{x \in \mathbb{R}^n} f(x)
\end{equation*}
where $f \in C_L^{1,1}(\mathbb{R}^n)$. We proved that $\frac{1}{L}$ was actually the best step length choice and that it guaranteed
\begin{equation}
\min_{0 \le i \le N} \vnorm{\nabla f(x_i)} \le \sqrt{\frac{2L(f(x_0)-f(x^*))}{N+1}} .
\label{MinGrad}
\end{equation}
Observe that this inequality
\begin{itemize}
\item is scaling independent,
\item doesn't say anything about the values of $f$.
\end{itemize}
One can show that this inequality is not improvable. \\

Let us recall the way we obtained the above inequality.

\begin{lemma}[Quadratic bounds in $C_L^{1,1}$]
The following conditions are equivalent :
\begin{enumerate}[label=(\alph*)]
\item $f \in C_L^{1,1}(D)$,
\item $f \in C^1(D)$ and $|f(y) - T_x^1(y)| \le \frac{L}{2}\vnorm{x-y}^2$ for all $x,y \in D$.
\end{enumerate}
\label{qbCL}
\end{lemma}
\begin{proof}
See the fourth exercises session.
\end{proof}

From this lemma, we concluded that $\frac{1}{L}$ is the optimal step length.

\begin{theorem}[Decrease guarantee]
Let $f \in C_L^{1,1}$. Denote $x^+ = x - \frac{1}{L}\nabla f(x)$ the next iterate. Then
\begin{equation*}
f(x) - f(x^+) \ge \frac{{\vnorm{\nabla f(x)}}^2}{2L} .
\end{equation*}
\label{DecreaseGuar}
\end{theorem}
\begin{proof}
Use the upper bound of lemma \ref{qbCL} with $y = x^+$.
\end{proof}
\noindent Actually there exist a family of functions which can be as closed of this bound as you want. \\

Finally theorem \ref{DecreaseGuar} leads to the inequality \eqref{MinGrad}.

\subsection{Gradient method for functions of $F_L^{1,1}$}

Let us now consider the same problem with the additional assumption $f \in F_L^{1,1}(\mathbb{R}^n)$. Lemma \ref{qbCL} can be improved as follows.

\begin{lemma}[Quadratic bounds in $F_L^{1,1}$]
The following conditions are equivalent :
\begin{enumerate}[label=(\alph*)]
\item $f \in F_L^{1,1}(D)$,
\item $f \in C^1(D)$ and $T_y^1(x) \le f(x) \le  T_y^1(x) + \frac{L}{2}\vnorm{x-y}^2$ for all $x,y \in D$.
\end{enumerate}
\label{qbFL}
\end{lemma}
\begin{proof}
See the fourth exercises session.
\end{proof}

\begin{lemma}
Let $f \in C_L^{1,1}$. For any optimal solution $x^*$ and any $x$,
\begin{equation*}
\frac{{\vnorm{\nabla f(x)}}^2}{2L} \le f(x) - f(x^*) \le \frac{L}{2} {\vnorm{x-x^*}}^2 .
\end{equation*}
\end{lemma}
\begin{proof}
The first inequality follows from theorem \ref{DecreaseGuar} since $f(x^+) \ge f(x^*)$. The second inequality follows from lemma \ref{qbCL} applied with $x = x^*$. Indeed since $f \in C^1$ and $x^*$ is a local extremum, we have $\nabla f(x^*) = 0$.
\end{proof}

\begin{theorem}[Convergence of $\frac{1}{L}$-gradient method for $F_L^{1,1}$]
Let $f \in F_L^{1,1}$. For any iterate $x_N$ and any $x$,
\begin{equation*}
f(x_N) - f(x^*) \le \frac{L}{2N} \vnorm{x_0-x^*}^2 .
\end{equation*}
\end{theorem}
\begin{proof}
We start from theorem \ref{DecreaseGuar}:
\begin{equation*}
f(x^+) \le f(x) - \frac{{\vnorm{\nabla f(x)}}^2}{2L} .
\end{equation*}
Since $f \in C^1$ and $f$ is convex, we have
\begin{equation*}
f(x^*) \ge f(x) + \nabla f(x)^T(x^*-x),
\end{equation*}
right-hand side being tangent equation around $x$. Combining these two inequalities yields
\begin{equation*}
f(x^+) \le f(x^*) + \nabla f(x)^T(x-x^*) - \frac{1}{2L} {\vnorm{\nabla f(x)}}^2 .
\end{equation*}
Now, observe that\footnote{Apply $\vnorm{a-b}^2 = \vnorm{a}^2 - 2a^Tb + \vnorm{b}^2$ to $a = x-x^*$ and $b = \frac{1}{L}\nabla f(x)$. Ok, it's a trick.}
\begin{equation*}
\nabla f(x)^T(x-x^*) - \frac{1}{2L}{\vnorm{\nabla f(x)}}^2 = \frac{L}{2} \left( \vnorm{x-x^*}^2 - \vnorm{x-x^*-\frac{1}{L}\nabla f(x)}^2 \right) .
\end{equation*}
Noting that $x-\frac{1}{L}\nabla f(x) = x^+$, we obtain
\begin{equation*}
f(x^+) - f(x^*) \le \frac{L}{2} \left( \vnorm{x-x^*}^2 - \vnorm{x^+-x^*}^2 \right) .
\end{equation*}
So given $N \in \mathbb{N}$, we have for all $i \in \{0,...,N-1\}$
\begin{equation*}
f(x_{i+1}) - f(x^*) \le \frac{L}{2} \left( \vnorm{x_i-x^*}^2 - \vnorm{x_{i+1}-x^*}^2 \right) .
\end{equation*}
Summing those $N$ inequalities yields
\begin{align*}
\sum_{i=0}^{N-1}f(x_{i+1}) - N f(x^*) &\le \frac{L}{2} \sum_{i=0}^{N-1}\left( \vnorm{x_i-x^*}^2 - \vnorm{x_{i+1}-x^*}^2 \right) \\
&= \frac{L}{2} \left( \vnorm{x_0-x^*}^2 - \vnorm{x_N-x^*}^2 \right) \\
&\le \frac{L}{2} \vnorm{x_0-x^*}^2 .
\end{align*}
Notice now that $f(x_N) \le f(x_i)$ for all $i \in \{0,...,N-1\}$ so that
\begin{equation*}
 N f(x_N) \le \sum_{i=1}^Nf(x_i) .
\end{equation*}
Using this in the last inequality, we finally get
\begin{equation*}
f(x_{N}) - f(x^*) \le  \frac{L}{2N} \vnorm{x_0-x^*}^2 . \qedhere
\end{equation*}
\end{proof}

Among all methods with
\begin{equation*}
x_k \in \myspan\{x_0, \nabla f(x_0), ..., \nabla f(x_{N-1})\},
\end{equation*}
none of them can guarantee better than
\begin{equation*}
f(x_N) - f(x^*) \le \frac{3}{32}L \frac{{\vnorm{x_0-x^*}}^2}{(N+1)^2}
\end{equation*}
for dimension greater or equal to $2N+1$.

\section{Gradient method for constrained problems}

We now add constraints. We consider problems of the following form:
\begin{equation*}
\min_{x \in C} f(x) \qquad \text{with} \quad f \in C_L^{1,1}(C)
\end{equation*}
and
\begin{equation*}
\min_{x \in C} f(x) \qquad \text{with} \quad f \in F_L^{1,1}(C) .
\end{equation*}
We assume $C \subseteq \mathbb{R}^n$ is a convex and closed set. This implies that the orthogonal projection on $C$
\begin{equation*}
P_C : \mathbb{R}^n \to C : x \mapsto P_C(x)
\end{equation*}
is well defined and unique.

\begin{defi}
Given $f \in C_L^{1,1}(C)$, we say that $x^*$ is a \emph{stationary point} of the problem $\min\limits_{x \in C} f(x)$ if and only if
\begin{equation*}
\nabla f(x^*)^T(x-x^*) \ge 0
\end{equation*}
for all $x \in C$.
\end{defi}
This definition can be intuitively interpreted as follows: adding $f(x^*)$ on both sides brings up the first-order Taylor expansion of $f$ around $x^*$, which is closed to $f(x)$. So this definition essentially means $f(x) \ge f(x^*)$.

Note that if $x^* \in \myint\, C$, then necessarily $\nabla f(x^*) = 0$.
Indeed, if $\nabla f(x^*) \neq 0$ and $x^* \in \myint\, C$, we can choose $x$ such that $\nabla f(x^*)$ and $x-x^*$ are of opposite directions. Consequently, their scalar product is negative which contradicts the definition of $x^*$. This implies that $\nabla f(x^*) = 0$.

\begin{theorem}
Under the above assumptions, if $x^*$ is a local minimum, then $x^*$ is stationary.
\end{theorem}

\begin{theorem}
When $f$ is convex, stationarity implies optimality.
\end{theorem}

Let us now present the gradient method for constrained problems. The principle is the following:
\begin{enumerate}
\item at each step, minimize the quadratic upper bound on set $C$,
\item which is equivalent to projecting the true minimum of the quadratic upper bound on set $C$.
\end{enumerate}
Let us show this equivalence. Statements mean
\begin{enumerate}
\item choose $x^+$ minimizing $f(x) + \nabla f(x)^T(x^+-x) + \frac{L}{2}{\vnorm{x-x^+}}^2$ over $C$, where we can ignore the constant term $f(x)$ in the minimization problem,
\item choose $x^+$ minimizing ${\vnorm{x^+ - (x - \frac{1}{L}\nabla f(x))}^2}$ over $C$. Notice that we can develop
\begin{equation*}
{\vnorm{x^+ - (x - \frac{1}{L}\nabla f(x))}^2} = {\vnorm{(x^+ - x) + \frac{1}{L}\nabla f(x))}^2} = \vnorm{x^+-x}^2 + \dfrac{2}{L}(x^+-x)^T \nabla f(x) + \dfrac{1}{L^2} \vnorm{\nabla f(x)}^2,
\end{equation*}
which is equivalent to 1 since we can ignore the constant term $\vnorm{\nabla f(x)}^2/L^2$ and multiply by $L/2$ without changing the minimization problem.
\end{enumerate}

This results in the following algorithm.
\begin{center}
\begin{tabular}{|l|}
\hline
Projected gradient method \\
\hline
\textbf{Given} $x_0,L,k=0$  \\
\textbf{Repeat} \\
\qquad $x_{k+1} = P_C(x_k - \frac{1}{L}\nabla f(x_k))$ \\
\qquad $k \leftarrow k+1$ \\
\hline
\end{tabular}
\end{center}

\end{document}
