
\documentclass[11pt]{scrreprt} %Type de doc. %scrartcl

% TEMPORARY
\usepackage{soul}


% standard
\usepackage[french,english]{babel}
\usepackage[utf8]{inputenc} %Caracteres accentués           
\usepackage[T1]{fontenc} 	%Police accentuée
\usepackage{lmodern}			%Police vectorielle (haute qualité)
\usepackage{vmargin} %Marges normales en A4
\usepackage{amsmath} %Insertion d'equations
\usepackage{todonotes}

\usepackage{amssymb}
\usepackage{vmargin} 

\usepackage{graphicx} %Images dans le PDF
\usepackage{float} 		%Flottants : Figures,Tables
\usepackage{color}		%Utilisation de couleurs
\definecolor{gris}{gray}{0.5}
\usepackage{eurosym}
\usepackage[hyphens]{url}
\usepackage{hyperref}
% \usepackage{breakurl}
\usepackage{wrapfig}
\usepackage{multirow} 
\usepackage{rotating}
\usepackage[small,bf]{caption}
\usepackage{textcomp}

\usepackage{subfig} 

\usepackage{amsthm}
\usepackage{pgf}

\usepackage{epstopdf}
\usepackage{attachfile} 
\usepackage{longtable}

\usepackage[framemethod=tikz]{mdframed}

\theoremstyle{plain}
\newtheorem{example}{{\bf Example}}[chapter]
\numberwithin{example}{chapter}
\newtheorem{theorem}{{\bf Theorem}}[chapter]
\numberwithin{theorem}{chapter}
% \newtheorem{property}[example]{{\bf Property}}
\newtheorem{definition}{{\bf Definition}}[chapter]
\numberwithin{definition}{chapter}
\newtheorem{lemma}{{\bf Lemma}}[chapter]
\numberwithin{lemma}{chapter}
\newtheorem{property}{{\bf Property}}[chapter]
\numberwithin{property}{chapter}
% \newtheorem{hypothesis}[example]{{\bf Hypothesis}}
% \newtheorem{exercise}{{\bf Exercise}}[section]
\newtheorem{thesis}{Thesis}[chapter]
\numberwithin{thesis}{chapter}
\theoremstyle{remark}
\newtheorem*{remark}{Remark}%[chapter]
% \numberwithin{remark}{chapter}

\surroundwithmdframed[outerlinewidth=2,roundcorner=10pt,leftmargin=0,rightmargin=0,% backgroundcolor=yellow!40,outerlinecolor=blue!70!black,
innertopmargin=\topskip,splittopskip=\topskip,ntheorem=true]{theorem}
\surroundwithmdframed[outerlinewidth=2,roundcorner=10pt,leftmargin=0,rightmargin=0,% backgroundcolor=yellow!40,outerlinecolor=blue!70!black,
innertopmargin=\topskip,splittopskip=\topskip,ntheorem=true]{lemma}
\surroundwithmdframed[outerlinewidth=2,roundcorner=10pt,leftmargin=0,rightmargin=0,% backgroundcolor=yellow!40,outerlinecolor=blue!70!black,
innertopmargin=\topskip,splittopskip=\topskip,ntheorem=true]{thesis}
 

\setcounter{secnumdepth}{3}
\setcounter{tocdepth}{3}

\newcommand{\E}{\mathbb{E}}
\newcommand{\R}{\mathbb{R}}

\allowdisplaybreaks[3] % ou x prend une valeur entière comprise entre 0 et 4; Plus x est gd, plus le compilateur accepte les coupures sur deux pages dans une formule.
% permet que les align puissent se prolonger sur pls pages

%========================================
\usepackage{listings}	%Inclusion de code-source
\lstset{							%Paramètres généraux pour les inclusions de code
	  language=matlab,
    basicstyle=\ttfamily\small,
    aboveskip={1.0\baselineskip},
    belowskip={1.0\baselineskip},
    columns=fixed,
    extendedchars=true,
    breaklines=true,
    tabsize=4,
    prebreak=\raisebox{0ex}[0ex][0ex]{\ensuremath{\hookleftarrow}},
    frame=lines,
    showtabs=false,
    showspaces=false,
    showstringspaces=false,
    keywordstyle=\color[rgb]{0.627,0.126,0.941},
    commentstyle=\color[rgb]{0.133,0.545,0.133},
    stringstyle=\color[rgb]{01,0,0},
    numbers=left,
    numberstyle=\small,
    stepnumber=1,
    numbersep=10pt,
    captionpos=t,
    escapeinside={\%*}{*)}
	}
\renewcommand{\lstlistingname}{\textsc{Algorithm}}	%Remplacer 'Listing' par 'Code' dans les légendes
%\usepackage[french,boxed,linesnumbered]{algorithm2e}	%Package algorithme avec options
\usepackage[english,algoruled, linesnumbered]{algorithm2e}	%Package algorithme avec options

\hypersetup{
colorlinks,%
citecolor=black,%
filecolor=black,%
linkcolor=black,%
urlcolor=black
} 

\makeatletter
\def\url@leostyle{%
  \@ifundefined{selectfont}{\def\UrlFont{\sf}}{\def\UrlFont{\small\ttfamily}}}
\makeatother
%% Now actually use the newly defined style.
\urlstyle{leo}


 \usepackage{layout}
% \usepackage{geometry}
 \usepackage{array}
 \usepackage{fancyvrb}
 \usepackage[parfill]{parskip} % espace apres paragraph
%% \usepackage{fourier} % not to use
\usepackage{pdfpages}
%
\newcommand{\QED}{\begin{flushright} QED \end{flushright}}
\newcommand{\ent}[1]{\lfloor #1 \rfloor}
\newcommand{\ceil}[1]{\lceil #1 \rceil}
\newcommand{\tup}[2]{\left( \begin{matrix} #1\\ #2\end{matrix} \right)}

% section numbering type
\renewcommand\thechapter{\Alph{chapter}}
\renewcommand\thesection{\arabic{section}}
\renewcommand\thesubsection{\hspace{0.5cm}\alph{subsection}}
\renewcommand\thesubsubsection{\hspace{1cm}\roman{subsubsection}}

\usepackage{framed}

\renewenvironment{proof}{{\bfseries Proof}}{\QED}

\usepackage{tocstyle}
\usetocstyle{standard}

\usepackage{tikz}
\usetikzlibrary{shapes,arrows,calc,decorations.pathmorphing,decorations.pathreplacing}
\tikzset{snake arrow/.style=
{->,
decorate,
decoration={snake,amplitude=.4mm,segment length=2mm,post length=1mm}},
}