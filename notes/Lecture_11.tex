%\documentclass[a4paper,11pt]{article}
%\usepackage[utf8]{inputenc}
%\usepackage[T1]{fontenc}
%\usepackage{lmodern}
%\usepackage{geometry} 
%\usepackage{graphicx}
%\usepackage{float}
%\usepackage{amsfonts}
%\usepackage{array}
%\usepackage{amsmath}
%\usepackage{mathtools}
%\usepackage{color}
%\usepackage{framed}
%\usepackage{caption}
%\usepackage{subcaption}
%\usepackage{hyperref}
%\usepackage{listings}
%\renewcommand{\bar}{\overline}
%
%\geometry{hscale=0.8,vscale=0.8}
%
%\title{LINMA2471 : Optimization models and models : course 11}
%\author{Maxime Duyck \\ Rémi Chauvenne}
%\date{December 2015}
%
%\begin{document}
%
%\maketitle

\subsection{Self-concordant function}
\begin{definition}\label{Def:1D} A function $f$ is called self-concordant if and only if
\begin{itemize}
\item $f \in \mathcal{C}^3(X)$ (open domain $X$)
\item $f$ convex
\item $\nabla^3 f(x) [h,h,h] \leq 2 (h^T \nabla^2 f(x)\ h)^{3/2} \quad \forall x \in X, \forall h$
\end{itemize}
\end{definition}

%\paragraph{Definition 1.} A function $f$ is self-concordant if and only if
%\begin{itemize}
%\item $f \in C^3(x)$ (on an open domain X)
%\item $f$ is convex
%\item $\nabla^3 f(x)[h,h,h] \le 2[h^T \nabla^2 f(x) h]^\frac{3}{2}$
%\end{itemize}
Let's define $F_{x,h}(t) : t \rightarrow f(x+th)$. We have
\begin{align*}
\nabla^3 f(x)[h,h,h] & = F'''_{x,h}(0) \stackrel{1D}{=} f'''(x)h^3 \\
\nabla^2 f(x)[h,h] & = F''_{x,h}(0) \stackrel{1D}{=} f''(x)h^2\\
\nabla f(x)[h] & = F'_{x,h}(0) \stackrel{1D}{=} f'(x)h\\
\end{align*}
where 1D means that we are studying the one-dimensional case. Let's look at the third condition of Definition \ref{Def:1D} in 1D :
\begin{align*}
f'''(x)h^3 & \le 2 (f''(x)h^2)^\frac{3}{2} \\
\end{align*}
which is equivalent to
\begin{align*}
\left\{
\begin{aligned}
f'''(x) \le 2f''(x)^\frac{3}{2} \quad \text{($h$ positive)} \\
-f'''(x) \le 2f''(x)^\frac{3}{2} \quad \text{($h$ negative)}
\end{aligned}
\right.
\end{align*}
Then one can say that in the one-dimensional case :
\begin{align*}
|f'''(x)| \le 2f''(x)^\frac{3}{2}
\end{align*}
for a self-concordant function.
\paragraph{Example 1.} $f(x) = -\log{(x)}$. 
\begin{align*}
f'(x) = -\frac{1}{x} \quad f''(x) = \frac{1}{x^2} \quad f'''(x) & = -\frac{2}{x^3} \\
\Rightarrow \left|-\frac{2}{x^3}\right| = 2 \left(\frac{1}{x^2}\right)^\frac{3}{2}
\end{align*}

\begin{property}
Let $f$ and $g$ be self-concordant (s.c.)\ functions. Then $f+g$ is a s.c.\ function.
\end{property}

\begin{example}
\begin{leftbar}
$$f(x) : \mathbb{R}_{++}^n \rightarrow \mathbb{R} : -\sum \log{(x_i)}$$ is s.c.\ as it is a sum of s.c.\ functions.
\end{leftbar}
\end{example}

\begin{property}
Let $x \rightarrow f(x)$ be a s.c. function. Then $y \rightarrow f(c-A^Ty)$ is s.c.\
\end{property}

\begin{example}
\begin{leftbar}
$$f(y) = - \sum\limits_{i=1}^m \log{(c_i-a_i^Ty)}$$ on domain $\{y|c-A^Ty \geq 0\}$ is s.c.\
\end{leftbar}
\end{example}

\begin{example}
\begin{leftbar}
$$f(x_0,x_1,...,x_n) = -\log{(x_0^2-x_1^2-...-x_n^2)}$$ is s.c. on int$(\mathbb{L}^n)$ with $\mathbb{L}^n = \left\{(x_0,x_1,...,x_n) | x_0 \geq \sqrt{x_1^2+...+x_n^2}\right\}$
\end{leftbar}
\end{example}

\begin{example}
\begin{leftbar}
Let $X \in \mathbb{S}^n$.
$$f(X) = -\log{\det{X}}$$
is s.c.\ on int$(\mathbb{S}_+^n)$
\end{leftbar}
\end{example}

\begin{property}
Let X be a set that contains no line. Then one can say that
\begin{itemize}
\item any s.c.\ function in X satisfies $\nabla^2 f(x) > 0$
\item $f(x) \rightarrow + \infty$ as $x \rightarrow \delta(X)$
\end{itemize}
where $\delta(X)$ is the boundary of set X.
\end{property}

\subsection{Minimisation of s.c.\ functions with Newton's method}
S.c.\ functions are easy to minimize with Newton's method. But the optimality measure given by $|| \nabla f(x) ||$ is bad as it is not affine-invariant. Let's try with another norm.
\begin{definition} \textbf{Local norm} (given a s.c.\ function f) : $(x \in X)$ 
$$
||z||_x  = (z^T \nabla^2f(x)z)^{\frac{1}{2}}
$$
The dual of the local norm is given by
$$
||z||^*_x  = (z^T \nabla^2f(x)^{-1}z)^{\frac{1}{2}}
$$
\end{definition}

Using the dual of the local norm, we have the optimality measure given by
\begin{align*}
\delta(x) &= ||\nabla f(x)||^*_x \\
& = (\nabla f(x)^T \nabla^2f(x)^{-1}\nabla f(x))^{\frac{1}{2}} \\
&= ||n(x)||_x
\end{align*}

\begin{property}
Given a s.c.\ function (on domain X), $x\in X$. If $\delta(x) < 1$ then 
\begin{itemize}
\item min $x^*$ of f exists
\item $f(x) \leq f(x^*) - \delta - \log{(1-\delta)} \approx f(x^*) - \frac{\delta^2}{2}$
\item $||x-x^*||_x \leq \frac{\delta}{1 - \delta} $
\item $x^+ = x + n(x)$ is feasible $(x^+ \in X)$
\item $\delta(x^+) \leq \left(\dfrac{\delta(x)}{1-\delta(x)}\right)^2 $\\
\end{itemize}
And for any $\delta(x)$, we have 
\begin{itemize}
\item $x^+ = x + \left(\dfrac{n(x)}{1+\delta(x)}\right) $ is feasible
\item $f(x) - f(x^+) \geq \delta(x) - \log{(1+\delta(x))} \geq 0 $
\end{itemize}
with $\left(\dfrac{1}{1+\delta(x)}\right) \le 1$ which is the step length.
\end{property}

\begin{example}
\begin{leftbar}
\textbf{(Application)}\\ 
Suppose $\delta(x_0) > \dfrac{1}{\sqrt{2}}$, we have $$f(x_0) - f(x_1) \geq 0.16.$$ 
If again $\delta(x_1) > \dfrac{1}{\sqrt{2}}$, then $$f(x_1) - f(x_2) \geq 0.16.$$ Hence, after at most $\dfrac{f(x_0) - f(x^*)}{0.16}$ iterations, we have $\delta(x) \leq \dfrac{1}{\sqrt{2}}$. \\
After applying $O(\log{\log{\frac{1}{\epsilon}}})$ pure Newton steps, we obtain an $\epsilon-$solution.
\end{leftbar}
\end{example}

%\end{document}
