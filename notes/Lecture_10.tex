%\documentclass[a4paper]{article}
%
%\usepackage[english]{babel}
%\usepackage[utf8]{inputenc}
%\usepackage{amsmath}
%\usepackage{amssymb}
%\usepackage{graphicx}
%\usepackage[colorinlistoftodos]{todonotes}
%\usepackage[left=2cm,right=3cm,top=2cm,bottom=1.3cm]{geometry}
%
%
%\title{LINMA2471: Optimization models and methods: course 11}
%\author{BROGNIET Adrien, DEFOUR Loïc, NINANE Charles}
%\date{December 1, 2015}
%
%\begin{document}
%\maketitle


\section{Conic modeling}
So far we have seen two types of cones: 
\begin{itemize}
\item $\mathbb{R}_n^+$ first order cone to represent linear constraints.
\item $\mathbb{L}^n$ second order cone to represent quadratics constraints like ratio, second and third power, ...
\end{itemize}
Theses cover a large class of problems but we will see a third type of cone. 
\subsection{Positive semi-definite cone}
A positive semi-definite cone is defined as: \{ M $\in \mathbb{S}^n | M \succeq 0 $ \} where $ \mathbb{S}^n$ is the set of symmetric matrices of size n. 
The following definitions of a positive semi-definite matrix are equivalent:
\begin{itemize}
\item $\mathbb{M} \succeq 0 $
\item All eigenvalues of M are greater than 0.
\item $\exists B \in \mathbb{R}^{n \times r}: M = BB^T  $ (kind of Cholesky factorisation)
\item $\forall x: x^T M x \geq 0$  
\end{itemize}

\subsubsection{Example conic optimization with semi-definite cone}
Find a symmetric matrix M such that:
\begin{itemize}
\item $ M = \begin{pmatrix}
   1 & 2 & 5\\
   2 & x & -1\\
   5 & -1 & y
\end{pmatrix}$ 
\item $\lambda_{max}$ is minimum. 
\end{itemize}
$\emph{Idea:}$
$\newline$
Let $M \in \mathbb{S}^n$ with eigenvalues $\lambda_1, \cdots, \lambda_n$. The eigenvalues are real because M is symmetric. It's easy to see that $ M + \Delta I \in \mathbb{S}^n$ has eigenvalues $ \lambda_1 + \Delta, \cdots, \lambda_n + \Delta $. By applying a shift on the diagonal of M we also shift the eigenvalues. By applying these observations to matrix M we can write the following \textcolor{red}{some of the $\geq$ or $\succeq$ seem to be inverted}: 
All eigenvalues of M $\geq 0 \Leftrightarrow M \succeq  0 \Leftrightarrow \lambda_{min}(M) \geq 0$
$\newline$
All eigenvalues of M $\geq 1 \Leftrightarrow M - I \succeq 0 \Leftrightarrow \lambda_{min}(M) \geq 1$
$\newline$
All eigenvalues of M $\geq c \Leftrightarrow M - cI \preceq 0$  (linear even if c is a variable)$  \Leftrightarrow \lambda_{min}(M) \geq c$
$\newline$
All eigenvalues of M $\leq c \Leftrightarrow M - cI \preceq 0   \Leftrightarrow \lambda_{max}(M) \leq c \Leftrightarrow - (M-cI) \succeq 0 = cI-M \succeq 0$ which is also a linear constraint.
$\newline$
$\newline$
We can now reformulate the optimization problem as:

$$ \min_{x,y,t} t \textnormal{ s.t } - \begin{pmatrix}
   1 & 2 & 5\\
   2 & x & -1\\
   5 & -1 & y
\end{pmatrix} + tI \succeq 0 $$ 

\chapter{Interior-point methods}	

\section{Pros and cons (of the Newton's method)}
\vspace{0.3cm}
\begin{tabular}{l|l}
\multicolumn{1}{c|}{Pros} & \multicolumn{1}{c}{Cons} \\ 
\hline
 & \\
- Faster convergence than first order methods & - More expensive computation \\
 & \\
- Potential quadratic convergence under specific conditions & - Not always well defined \\
 & \\
 & -  Not globally convergent 
\end{tabular}

\section{Computing the Newton's method}
We consider the unconstrained optimization program :
$$ \underset{x}{\min} \hspace{0.1cm} f(x) $$
$$ x \in \mathbb{R}^n $$
We use the second order Taylor expansion to write :
$$ f(x) = f(x_k + h) \approx f(x_k) + \nabla f(x_k)^T h + \frac{1}{2} h^T\nabla^2 f(x_k) h$$
with $x_k$ a fixed point and h a step such that $x_k,h \in \mathbb{R}^n $. We know that the stationary points of the function $f$ are reached when $\frac{ \partial f(x_k+h)}{\partial h} = 0$. Regarding the previous approximation, we have :
$$\frac{ \partial f(x_k+h)}{\partial h} = \nabla f(x_k) + \nabla^2 f(x_k) h $$
Therefore we can choose the step $h$ such that it satisfies the linear system :
$$\nabla^2 f(x_k) h = -\nabla f(x_k) $$
Assuming $\nabla^2 f(x_k)$ is invertible, we have :
$$ h = -\nabla^2 f(x_k)^{-1} \nabla f(x_k)$$
The Newton's method is then:

\begin{lstlisting}[mathescape,caption=Newton Algorithm]
Given $x_0$, $f$, $\nabla f$, $\nabla^2 f$ invertible, $k=0$
Repeat
$x_{k+1} = x_k - \nabla^2 f(x_k)^{-1} \nabla f(x_k) $
$k \leftarrow k+1$
\end{lstlisting}

\begin{example}
\begin{leftbar} 
Using the Newton's method, we can find the global minimum of the function $f(x) = x^2$ in one iteration $ \Longrightarrow x^+ = x - (2)^{-1}(2x) = 0$.
\end{leftbar}
\end{example}

\begin{example}
\begin{leftbar} 
Let us apply the Newton's method on $f(x) = -\cos(x)$. \\
We get that $x^+ = x - (\cos(x))^{-1}(\sin(x)) = x - \tan(x) = x - (x + \mathcal{O}(x^3))$ (near to 0, quadratic convergence) 
\end{leftbar}
\end{example}

\begin{example}
\begin{leftbar} 
Let us apply the Newton's method on $f(x) = x^4$. \\
We get that $x^+ = x - (12x^2)^{-1}(4x^3) = x - \frac{x}{3} = \frac{2}{3}x$. \\
We have a linear convergence.  \\
Why is that ? $f(x)$ has a "flat" minimum $\Longrightarrow \mbox{ in } x^* = 0, \nabla^2f(x^*) = 12x^* = 0 \nsucc \mu$.
\end{leftbar}
\end{example}

\subsection{Conditions for quadratic convergence}
\begin{theorem}{(Quadratic convergence)}Let $f \in \mathcal{C}^2$. If $\nabla^2f$ is $M$-Lipschitz and $x^*$ is a minimum of $f$ such that $\nabla^2f(x^*) \succeq \mu I$ (with $\mu \in \mathbb{R}$ and $I$ the identity matrix), then for any $x$ such that $||x-x^*|| \leq \frac{\mu}{2M}$ we have :
$$ ||x^+-x^*|| \leq \frac{M}{\mu} ||x-x^*||^2 \hspace{0.3cm}$$
with $x^+ = x - \nabla^2 f(x)^{-1} \nabla f(x)$ well defined.
\end{theorem}

\begin{example}
\begin{leftbar}
$\frac{\mu}{2M} \longmapsto \frac{M}{\mu} \cdot \left(\frac{\mu}{2M}\right)^2 = \frac{\mu}{4M} \longmapsto \frac{M}{\mu} \cdot \left(\frac{\mu}{4M}\right)^2 = \frac{\mu}{16M}$ (quadratic convergence) \\
After $k$ steps, we get $\frac{\mu}{M} \cdot \frac{1}{2^{2^{k-1}}}$.
\end{leftbar}
\end{example}

\subsection{Change of variables}
We have seen that an iteration of the Newton's method is defined by $x \longmapsto x - \nabla^2f(x)^{-1}\nabla f(x)$. \\
Let us define $x$ such that $x \triangleq Ay$ and $g(y) = f(Ay)$. \\ \\
We claim that $x^+ = Ay^+$ (hence : $f(x^+) = g(y^+))$. \\
Let us proof that claim. We compute that
$$ \nabla g(y) = A^T \nabla f(Ay), \qquad \nabla^2 g(y) = A^T \nabla^2 f(Ay) A$$
So we have \\
$y^+ = y - \nabla^2 g(y)^{-1}\nabla g(y) \Longrightarrow y^+ = y - (A^T \nabla^2 f(Ay) A)^{-1} A^T \nabla f(Ay) \Longrightarrow y^+ = y - A^{-1} \nabla^2 f(Ay) A^{-T} A^T \nabla f(Ay)$ \\ \\
If we multiply the last expression by $A$ on both sides, we get \\
$ Ay^+ = Ay - AA^{-1} \nabla^2 f(Ay) \nabla f(Ay) \Longrightarrow Ay^+ = Ay - \nabla^2 f(Ay) \nabla f(Ay) \Longrightarrow x^+ = x - \nabla^2 f(x)^{-1} \nabla f(x)$ \\ \\
We have shown that Newton's method is affine invariant.


% \end{document}
